%Dies ist Datei exercise_1/exercise_1.tex
\documentclass{article}
\usepackage{silence}
\WarningFilter{latex}{You have requested package}
\WarningFilter{latex}{Empty bibliography}
\WarningFilter{latex}{Interword spacing}
\newcommand{\tutor}{Andreas Rydh}
\newcommand{\modul}{condensed matter}
\newcommand{\sheet}{1}
\newcommand{\sheetauthor}{Elin Bayer}
\usepackage{../style}
\begin{document}
\section*{\#5.1}
\begin{itemize}
    \item[Given:] Free el. approx
    \item[Wanted:] Mean free path of cond. el. in Ag at room temp.
\end{itemize}

\subsection*{Solution}
The mean free path is given by 
\begin{align}
    l &= v_F \tau
    \intertext{Use \(\varepsilon_F = \frac{m\nu_F^2}{2}\) with that we get an expression for the Fermi velocity}
    v_F &= \sqrt{\frac{2\varepsilon_F}{m}} 
    \intertext{Ag has fcc structure and therefore 4 atoms per unit cell, this implies \(\frac{N_{At}}{V} = \frac{4}{a^3}\), still have to look up \(a\), we get \(\tau\) via the resistivity \(\rho = \frac{m}{nq^2\tau}\), the values are:}
    \rho_{Ag} &= \SI{10.5}{\gram\per\centi\meter\cubed}, \quad m = \SI{107.9}{\gram\per\mole}, \quad \rho = \SI{1.51e-6}{\ohm\centi\meter}, \quad a = \SI{4.09}{\angstrom}
\end{align}

\section*{\#5.2}
\begin{itemize}
    \item[Given:] CuZn alloy, \(\frac{N_{at}}{V}\) equal to \(\frac{N_{st}}{V}|_{Cu} \)
    \item[Wanted:] \(\varepsilon_F\) 
\end{itemize}

\subsection*{Solution}
The fermi energy is given by \(\varepsilon_F = \frac{\hbar^2k_F^2}{2m}\), with \(k_F = \left(3\pi^2n\right)^{1/3}\) and \(n = \frac{N_{At}}{V}\), we get \(\varepsilon_F = \frac{\hbar^2}{2m}\left(3\pi^2\frac{N_{At}}{V}\right)^{2/3}\). Since \(\frac{N_{At}}{V} = \frac{N_{st}}{V}|_{Cu}\) we can use the values for Cu to get the Fermi energy.
Cu has the structure fcc and therefore 4 atoms per unit cell, this implies \(\frac{N_{At}}{V} = \frac{4}{a^3}\) and for CuZn we have \(n= 1.5\cdot n_{Cu} \), where \(a_{Cu} = \SI{3.61}{\angstrom}\).

\section*{\#5.3}
\begin{itemize}
    \item[Given:] 3D \(\varepsilon_F = \frac{\hbar^2k^2_F}{2m}\) with \(k_F = \left(3\pi^2n\right)^{1/3}\)
    \item[Wanted:] corresponding 2D case for \(\varepsilon_F\) k
\end{itemize}

\subsection*{Solution}
Boundery condition: \(\psi(0) = \psi(L) = \frac{1}{A} e^{ik_x}\), with \(k_x\cdot L = 2\pi n_x\) and  therefore \(k_x = 2\pi n_x/L\).
\begin{align}
    \frac{\pi k_F^2}{\frac{2\pi}{L}^2} \cdot 2 = N \implies k_F = \sqrt{2\pi \frac{N}{A}}
\end{align}

\section*{\#1.6}
\begin{itemize}
    \item[Given:] Si, density \(\rho = \SI{2.33}{\gram\per\centi\meter\cubed}\), "hard sphere"
    \item[Wanted:] If atoms arranged into close packing what is the density?
\end{itemize}

\subsection*{Solution}
Si has diamond structure

\end{document}

